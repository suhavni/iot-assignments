\documentclass[12pt]{article}

\usepackage{amsmath}

\usepackage{graphicx}

\usepackage{hyperref}

\usepackage[utf8]{inputenc}

\title{IOT Assignment 1}

\author{Pornkamol Luthra}

\date{10 Dec 2022}

\input{arduinoLanguage.tex}

\begin{document}

\maketitle

\section*{Task 1: Disco Lights}

\begin{enumerate}
	\item YouTube Link: \href{https://youtu.be/KJ-E_A0_tmo}{https://youtu.be/KJ-E\_A0\_tmo}
	\item Doodle \\\\
		\includegraphics[scale=0.15]{task1/sketch.jpg}
	\item Code
		\begin{itemize}
			\item GitHub Link:\\ 
				\href{https://github.com/suhavni/iot-assignments/blob/main/HW1/task1/task1.ino}{https://github.com/suhavni/iot-assignments/blob/main/HW1/task1/task1.ino}
			\item Code Snippet:
				\begin{lstlisting}[language=Arduino]
		unsigned long startedOn = 0;
		long interval = 1;
		int currentLED = 12;
		int nextLED = 2;

		void setup() {
			startedOn = millis();
			for (int i = 2; i <= 12; i+=2) {
				pinMode(i, OUTPUT);
			}
		}

		void loop() {
			int sensorValue = analogRead(A0);
			
			unsigned long currentMillis = millis();

			nextLED = (sensorValue < 512) ? -2 : 2;
			sensorValue = (sensorValue < 512) ? sensorValue : 1023 - sensorValue;
			interval = (sensorValue+15)*2;
			
			if(currentMillis - startedOn > interval) {
				startedOn = currentMillis;
				digitalWrite(currentLED+2, LOW);
				currentLED = (currentLED + nextLED + 12)%12;
				digitalWrite(currentLED+2, HIGH);
			}
		}
				\end{lstlisting}
	\end{itemize}
\end{enumerate}

\section*{Task 2: Ohmmeter}
\subsection*{2.1 Ohm's Law Review}
\includegraphics[scale=0.15]{task2/review1.jpg}

\newpage
\includegraphics[scale=0.15]{task2/review2.jpg}
\newpage

\subsection*{2.2 Building Ohmmeter}

\begin{enumerate}
	\item YouTube Link: \href{https://youtu.be/ia95OZDOZOQ}{https://youtu.be/ia95OZDOZOQ}
	\item Code
		\begin{itemize}
			\item GitHub Link:\\ 
				\href{https://github.com/suhavni/iot-assignments/blob/main/HW1/task2/task2.ino}{https://github.com/suhavni/iot-assignments/blob/main/HW1/task2/task2.ino}
			\item Code Snippet:
				\begin{lstlisting}[language=Arduino]
		#define R1 1000
		#define V_IN 5
		
		int sensorValue = 0;
		float Vout = 0;
		float R2 = 0;
		float buffer = 0;
		
		void setup() {
				Serial.begin(9600);
		}
		
		void loop() {
				sensorValue = analogRead(A0);
				if (sensorValue) {
						Vout = (sensorValue * V_IN)/1023.0;
						R2 = R1 * ((V_IN/Vout) - 1);
						Serial.print("Resistance: ");
						Serial.print(R2);
						Serial.println("Ohms");
						delay(1000);
				}
		}
				\end{lstlisting}
	\end{itemize}
\end{enumerate}

\newpage
\section*{Task 3: Coin Counter}
\begin{enumerate}
	\item YouTube Link: \href{https://youtu.be/smF-OGToIHc}{https://youtu.be/smF-OGToIHc}
	\item Sketch \\\\
		\includegraphics[scale=0.15]{task3/sketch.jpg}
	\item Code
		\begin{itemize}
			\item GitHub Link:\\ 
				\href{https://github.com/suhavni/iot-assignments/blob/main/HW1/task3/task3.ino}{https://github.com/suhavni/iot-assignments/blob/main/HW1/task3/task3.ino}
			\newpage
			\item Code Snippet:
				\begin{lstlisting}[language=Arduino]
		int coins = 0;

		void setup() {
			Serial.begin(9600);
		}
		
		void loop() {
			int sensorValue = analogRead(A0);
			if (sensorValue && sensorValue < 200) {
				Serial.print("Coins: ");
				Serial.println(++coins);
				delay(30);
			}
		}
				\end{lstlisting}
	\end{itemize}
\end{enumerate}

\end{document}